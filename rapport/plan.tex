\section*{Hvordan vi har tenkt å gjøre prosjektet}
Denne rapporten er en del av vurderingen i matematikkdelen av faget TDAT3024. I dette prosjektet kommer vi til å se på rotasjon av stive legemer, spesifikt rotasjonen til en t-nøkkel som er modellert som to sylindere loddet sammen. Vi skal lage en modell og gjøre beregnigner for å bedre forstå hvordan t-nøkkelen vil oppføre seg i denne situasjonen.\newline\newline
Vi tenker å bruke primært Python 3 med bibliotekene NumPy and SciPy til å løse oppgavene. Dette fordi vi er mest vant til å bruke Python for numerisk utregning og føler det er lettere å kombinere Python med andre elementer vi også vil inkludere i vår løsning av prosjektet. For å løse oppgavene skal vi bruke teorien vi lærer i rotasjonsmekanikk i fysikkdelen og kombinere det med matematikkdelen. Det hadde vært interessant å sammenligne de ulike numeriske metodene og se hvilke resultater de gir. Vi tenker også å visualisere det stive legemet i 3D-grafikk.\newline\newline
De ulike numeriske metodene vi kommer til å bruke er:
\begin{itemize}
    \item Eulers Metode
    \item Midtpunkts-metoden
    \item Trapes-metoden
    \item RK
    \item RKF
\end{itemize}
I resultatdelen vil vi sammenligne feilen i de ulike numeriske metodene, og finne ut av hvilken metode som beskriver T-nøkkelen best.\newline\newline
Vi tenker også å bevise treghetsmomentet i nøkkelen med utgangspunkt i teorien fra fysikkdelen av faget. Her snakker vi om likningene i kapittel $7$ i prosjektbeskrivelsen.\newline\newline
For å animere rotasjonen tenker vi å bruke PyOpenGL ettersom vi har brukt OpenGL i tidligere kurs, skrive alt i Python. I tilleg kan det bli enklere å implementere en dynamisk modell slik at vi kan justere parametere på modellen under kjøring.\newline\newline
Ettersom det viktigste med prosjektet er at alle lærer av det, så tenker vi å jobbe sammen på hver oppgave istedenfor å dele opp arbeidet. På denne måten får alle gjennomgått hele teorien og arbeidet vi skal utføre.\newline\newline
Siden vi er dataingeniører, så tenker vi å bruke versjonskontroll, nærmere bestemt Git. Dette er for å synkronisere arbeidet vårt. Med dette kan vi holde både rapportskriving og kode samlet og kontrollert. Vi skriver rapporten i Overleaf for å enklere få god struktur og har koblet Overleaf med Github. Dette gjør at endringer vi gjør på rapportskrivingen lett kan synkroniseres med resten av arbeidet som blir gjort. \newline\newline
En annen ting som hadde vært interessant å se på er å modifisere tyngdeakselerasjonen for å se hvordan det påvirker rotasjonen til det stive legemet.