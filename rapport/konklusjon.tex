\section{Konklusjon}
I dette prosjektet har vi undersøkt hvordan vi kan simulere t-nøkkelen ved hjelp av ulike numeriske metoder. Vi har brukt Eulers metode, RK4 og RKF45 til å tilnærme løsningene av differensiallikningene som definerer bevegelsen til nøkkelen. Vi har også funnet en eksakt løsning til rotasjon om en akse når det stive legemet er en kule. I tillegg til det har vi beregnet energien i systemet, og sett at den er relativt stabil. Dette kan vi se fra figur \ref{fig:energi}.\newline\newline
Vi har beregnet feilestimater for modellene våre som tilsier at modellene er ganske nøyaktige. I tillegg har vi lagd en animasjon hvor vi simulerer bevegelsen til t-nøkkelen. Ved å sammenligne vår simulasjon med konkrete eksempeler fra virkeligheten, så ser vi at vår modell beveger seg slik som forventet. Vi har også skrevet et kort bevis på hvorfor Mellomakse-teoremet er sann, og brukt denne matematikken til å forklare hvorfor t-nøkkelen har en ustabil rotasjon om $x$-aksen. Basert på disse observasjonene kan vi konkludere med at modellene våre stemmer ganske godt og det er en god simulering av Mellomakse-teoremet.\newline\newline
I dette prosjektet har vi lært om rotasjon av stige legemer og hvordan treghetsmomentet til et legeme påvirker bevegelsen. Vi har også gått mer i dybden på rotasjon, da blant annet ved å studere mellomakseteoremet. Den sier at legemer ikke roterer stabilt om aksen med mellomstor treghet, det er derfor vi ser den underlige rotasjonen til t-nøkkelen. Siden rotasjonen ikke er stabil hopper den frem og tilbake.\newline\newline
I tillegg har vi sett nærmere på numeriske metoder. Dette har vi gjort gjennom å implementere og simulere flere forskjellige metoder og sammenlignet dem. Ved å gjøre dette har vi kunnet finne ut hvilke metoder som gir oss den beste modellen. I vår simulasjon fikk vi litt forskjellige resultater mellom de numeriske metodene. RK4 og RKF45 fikk til enhver tid resulter som var så like at forskjellene er neglisjerbare. Euler derimot fikk alltid forskjellig resultater fra de høyere ordens metodene. Når det gjelder resultatene i oppgave 5 bevarer Euler energien bedre i oppgave a), dårligere i oppgave b) og bedre igjen i oppgave c). Dette tyder på at nøyaktigheten ikke er direkte bedre for RK4 og RKF45, men kan godt være bedre hos Euler i noen tilfeller. Ettersom RK4 og RKF45 i resultatene virker å være like eller av og til mindre nøyaktige enn Euler, kan vi fort konkludere med at Euler er det beste og raskeste alternativet for simulasjonene.\newline\newline