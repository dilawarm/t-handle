\section{Erklæring}
\subsection{Tobias Meyer Andersen}
På starten av prosjektet var jeg med å implementere funksjonene i oppgave 1 og jeg skrev testene, jeg var også med å effektivisere og finne feil i implementasjonen av RK4 og RK45. Jeg har bidratt på plottingen av energien i matplotlib, samt dokumentere alle funksjonene vi har skrevet på en tydelig måte. På rapporten har jeg bidratt til å skrive  inn blant annet oppgave 1, oppgave 3, innledning, og konklusjon. I tillegg har jeg vært med å løse flere av oppgavene.
\vspace*{\fill}

\noindent\begin{tabular}{@{}p{2.5in}p{2.5in}@{}}
\dotfill                         & \dotfill\\
Kjerand Evje              & Dato\\
                                 & \\[8ex]
\dotfill                         & \dotfill\\
Dilawar Mahmood              & Dato\\
                                 & \\[8ex]
\dotfill                         & \dotfill\\
Håvard Stavnås Markhus              & Dato\\
                                 & \\[8ex]
\end{tabular}
\newpage
\subsection{Kjerand Evje}
Jeg har bidratt til prosjektet gjennom forskjellige oppgaver. Alle på gruppen samarbeidet om å løse oppgavene og skrive koden for dem. I tillegg har jeg bidratt på teori-delen hvor jeg skrev hele seksjonen om numerikk. Jeg har også bidratt med å skrive på "Diskusjon", "Konklusjon" og "Innledning" delene på rapporten, samt å sette opp nomenklatur.
\vspace*{\fill}

\noindent\begin{tabular}{@{}p{2.5in}p{2.5in}@{}}
\dotfill                         & \dotfill\\
Tobias Meyer Andersen              & Dato\\
                                 & \\[8ex]
\dotfill                         & \dotfill\\
Dilawar Mahmood              & Dato\\
                                 & \\[8ex]
\dotfill                         & \dotfill\\
Håvard Stavnås Markhus              & Dato\\
                                 & \\[8ex]
\end{tabular}
\newpage
\subsection{Dilawar Mahmood}
På dette prosjektet har jeg bidratt til å sette opp GitHub og Overleaf, og koble disse for å gjøre arbeidsflyten effektiv. Jeg har sammen med de andre på gruppa skrevet kode for de ulike oppgavene. Etter at vi ble ferdig med oppgavene, så har jeg rettet opp feil i kildekoden, nærmere bestemt feil i RKF45 og energiberegningene. Jeg skrev også punktgenereringen for å gjøre animasjonen av t-nøkkelen effektiv. Ellers har jeg hjulpet til litt med animasjonen av t-nøkkelen. Jeg skrev også store deler av oppgave 2, og jeg skrev hele teoridelen om rotasjon av stive legemer. Utenom det har jeg også bidratt til å skrive nomenklatur, oppgave 5, diskusjon og konklusjon. Jeg har også brukt mye tid på å sette meg inn i Mellomakse-teoremet, og skrev også et bevis for hvorfor det teoremet stemmer. I tillegg til det har jeg sørget for at koden har blitt dokumentert på en god måte og at den fungerer ordentlig, og også tatt ansvar for at strukturen til rapporten skal være god.
\vspace*{\fill}

\noindent\begin{tabular}{@{}p{2.5in}p{2.5in}@{}}
\dotfill                         & \dotfill\\
Tobias Meyer Andersen              & Dato\\
                                 & \\[8ex]
\dotfill                         & \dotfill\\
Kjerand Evje              & Dato\\
                                 & \\[8ex]
\dotfill                         & \dotfill\\
Håvard Stavnås Markhus              & Dato\\
                                 & \\[8ex]
\end{tabular}
\newpage
\subsection{Håvard Stavnås Markhus}
Under utføring av selve oppgavene har jeg, og alle de andre jobbet samtidig, slik at vi alle kunne lære oss hvordan selve oppgavene ble utført. Jeg har jobbet mye med animasjonen av t-nøkkelen, slik at disse ble tegnet riktig. Etter at vi ble initielt ferdig med oppgavene har jeg vært med på å legge inn forbedringer på de numeriske metodene ved å sjekke energinivået ved hver iterasjon, legge til rette parametere (slik at det ble enklere å justere steglengde) og rettet opp feil i forskjellige oppgaver. I rapporten har jeg bidratt med å skrive inn resultater fra oppgave 4, hjulpet til for resultater av oppgave 2, skrevet i diskusjonen når det gjelder animasjon og hjulpet til på konklusjon. Jeg har også brukt mye tid på å lage grafer og sørge for at koden fungerer som den skal.
\vspace*{\fill}

\noindent\begin{tabular}{@{}p{2.5in}p{2.5in}@{}}
\dotfill                         & \dotfill\\
Tobias Meyer Andersen              & Dato\\
                                 & \\[8ex]
\dotfill                         & \dotfill\\
Kjerand Evje              & Dato\\
                                 & \\[8ex]
\dotfill                         & \dotfill\\
Dilawar Mahmood              & Dato\\
                                 & \\[8ex]
\end{tabular}
\newpage