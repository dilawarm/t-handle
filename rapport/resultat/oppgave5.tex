\subsection{Oppgave 5}
\label{section:oppgave5}
I oppgave 5 har vi benyttet koden vi skrev i oppgave 4 og løst systemet med RKF45. \newline\newline
For å finne treghetsmomentet har vi benyttet kode fra oppgave 1 som ligger i "kode/oppgave1/oppgave1funksjoner.py" og masse, radius og lengde oppgitt tidligere i oppgaven. Videre har vi kalkulert $\vec{L}$ med en funksjon som også finnes i "kode/oppgave1/oppgave1funksjoner.py". For å regne ut $\vec{L}$ i denne funksjonen bruker vi likning (\ref{eq:dreieimpuls}) sammen med initialverdiene for $X$ og $\omega.$
\begin{equation}
\vec{L} = X_0I\vec{\omega_0}
\end{equation}
Siden $\vec{\omega_0}$ allerede har lenge ca. lik 1 i oppgavene a, b, og c, så regner vi $\vec{L}$ ut fra den. Dermed får vi en $\vec{L}$ slik at $\vec{\omega}$ har lengde ca. lik 1 ved tiden $t=0.$\newline\newline
For å kontrollere steglengden slik at energien ikke endrer seg nevneverdig, så har vi implementert dynamisk steglengde. Det vil si at hvis energiforskjellen mellom $t_0$ og $t_{i}$ er større enn en satt grense, som vi har valgt å sette lik $1\text{ }\mu\text{J}$, så vil steglengden halveres. Denne grensen er satt i \ref{kode:utils}, og denne formen for dynamisk steglengde er implementert for alle de numeriske metodene vi har brukt i prosjektet. Resultatene vi viser i deloppgavene blir alle regnet ut med RKF45, og steglengden er $2\cdot10^{-5}.$ med $0\leq t \leq 2$
\subsubsection{Deloppgave a}
Vi kjører RKF45 for $\vec{\omega}(0)=\begin{bmatrix}1&0.05&0\end{bmatrix}$ og får dette resultatet ved $t=2\text{ s}:$
\begin{equation}
X(2)=
\begin{bmatrix}
0.99610525 &  0.06248594 & 0.06222472\\
0.08260984 & -0.41480051 & -0.90628396\\
-0.03081946 & 0.90807585 & -0.41834688\\
\end{bmatrix}
\end{equation}
\subsubsection{Deloppgave b}
Her kjører vi RKF45 for $\vec{\omega}(0) = \begin{bmatrix}0 & 1 & 0.05 \end{bmatrix}$ og får dette resultatet ved $t=2\text{ s}:$

\begin{equation}
    X(2) = \begin{bmatrix}
    -0.41492445 & -0.11658854 & 0.90266841 \\
    -0.0079576 & 0.99218717 & 0.12451836 \\
    -0.90994987 & 0.04444193 & -0.41261256 \\
    \end{bmatrix}
\end{equation}
\subsubsection{Deloppgave c}
Vi kjører RKF45 for $\vec{\omega}(0)=\begin{bmatrix}0.05&0&1\end{bmatrix}$ og får dette resultatet ved $t=2\text{ s}:$
\begin{equation}
X(2)=
\begin{bmatrix}
-0.41759919 & -0.90741052 & 0.04952559\\
0.90889721 & -0.4175031 & 0.01276843\\
0.00909854 & 0.05032831 & 0.99869159\\
\end{bmatrix}
\end{equation}