\subsection{Oppgave 1}
\subsubsection{Deloppgave a}
Første oppgave går ut på å implementere funksjonen til ligning 21, nemlig: 
\begin{equation}
    exp(h\Omega) = \textit{I} + (1 - \cos(\omega h))\frac{\Omega^2}{\omega^2}+\sin(\omega h)\frac{\Omega}{\omega}
\end{equation}
Funksjonen skal altså ta inn steglengden h, og $\Omega$, en matrise som inneholder de ulike vinkelhastighetene. Dette ble implementert i python ved hjelp av numpy i [[[SKRIV INN HVA KODEFILEN HETER HER]]]. For å verifisere at funksjonen fungere som forventet så sjekker vi egenskapen at outputmatrisen X har egenskapen $X^T X = I$. Ettersom beregningene skjer numerisk på en datamaskin så testet vi om alle elementene i matrisen var mindre enn $10^{-4}$ unna den forventete verdien. Med testene bestått for verdien av flere ulike størrelsesordener kan vi være ganske sikre på at den er implementert riktig.
\subsubsection{Deloppgave b}
Neste deloppgave er å implementere funksjonen som beregner energien til et roterende legeme som har et gitt treghetsmoment og vinkelhastighetsvektor $\Vec{\omega}$. Dette ble implementert i python ved å løse følgende ligning for $K$, som er kinetisk energi.
\begin{equation}
    \textit{K} = \frac{1}{2}\vec{\textit{L}}\cdot \vec{\omega}
\end{equation}
Vi beregner dreieimpulsen $\vec{L}$ på følgende måte
\begin{equation}
\label{eq:dreieimpuls}
    \vec{L}=XI\vec{\omega_b}
\end{equation}
Implementasjonen [[[LIGGER PÅ DENNE PLASSEN SOM VI MÅ BESKRIVE SEINERE]]]
\subsubsection{Deloppgave c}
I siste deloppgave skal treghetsmomentet til nøkkelen beregnes, her skrev vi en funksjon som løser de følgende likningene gitt massene M, lengdene L, radiusene R til de to sylinderene som utgjør T-nøkkelen. \begin{equation}
    I_{xx}= \frac{M_1{R_1}^2}{4} + \frac{M_1{L_1}^2}{12} + \frac{M_2{R_2}^2}{2}
\end{equation}
\begin{equation}
    I_{yy}= M_1{R_1}^2 + \frac{M_2{L_2}^2}{4} + \frac{M_1{R_1}^2}{2} + \frac{M_2{R_2}^2}{4} + \frac{M_2{L_2}^2}{12}
\end{equation}
\begin{equation}
    I_{zz}= M_1{R_1}^2 + \frac{M_2{L_2}^2}{4} + \frac{M_1{R_1}^2}{4} + \frac{M_1{L_1}^2}{12} + \frac{M_2{R_2}^2}{4} + \frac{M_2{L_2}^2}{12}
\end{equation} 
\newline
de resterende elementene i matrisen er 0. Selve koden som gjør beregninger å finne i [[[HALLA KOMPIS, HER SKAL DU SKRRRRRRRIVE HVOR VI KAN LESE NEVNTE KODE]]]. Når koden kjøres får man følgende matrise som beskriver treghetsmomentet: \newline
\begin{equation}
I = 
\begin{bmatrix}
982.27130302 & 0 & 0\\
0 & 722.67103008 & 0\\
0 & 0 & 1578.65030843\\
\end{bmatrix}
\end{equation}