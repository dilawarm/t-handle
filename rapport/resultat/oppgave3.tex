\subsection{Oppgave 3}
\label{section:oppgave3}
I oppgave 3 har vi skrevet kode som anvender følgende variant av eulers metode for å løse difflikningen fra oppgave 2:

\begin{equation}
    W_0 = y(t_0)
\end{equation}

\begin{equation}
    W_{i+1} = W_i + hf(t_i, w_i)
\end{equation}

Ved å gjennomføre 10000 steg på et 2 sekunders intervall får numeriske metode følgende resultat:\newline
\begin{equation}
W_{10000} \approx    
\begin{bmatrix}
1 & 0 & 0\\
0 & -0.41623003 & -0.90938838\\
0 & 0.90957028 & -0.41623003\\
\end{bmatrix}
\end{equation}

Det stemmer veldig godt overens med den eksakte løsningen som er vist under:\newline
\begin{equation}
X(2) =     
\begin{bmatrix}
1 & 0 & 0\\
0 & -0.41614684 & -0.90929743\\
0 & 0.90929743 & -0.41614684\\
\end{bmatrix}
\end{equation}
Her har vi brukt $X(0)=I=Id_{3\times3}$ og $L=\begin{bmatrix}1&0&0\end{bmatrix}.$